\section{Access Control Module}

\subsection{Problem Summary}

\hspace{0.35cm} Currently, the complete content offered in the courses on Bodhitree can be accessed by any student who is registered for the course. If a student has access to a course, the he has access to all the videos, documents, quizzes and assignments in that course.
\par There might be a case where the instructor wants to limit the access of content to the students. He might want some students to not have access to a particular content type, for example, the instructor would want only those students to have access to the in-video quizzes who have made some payment.
\par The access control module enables the instructor to provide differential access to students based upon the access that he grants them.

\subsection{Specifications}

The specifications were provided by the content developers which are meant to control the access that each individual student has to the content offered on Bodhitree. They are as follows:

\begin{enumerate}
	\item The instructor can tag the elements present in each concept. These elements can be any one of the following:
	\begin{enumerate}
		\item Videos
		\item Documents
		\item Quizzes
		\item Video Markers (In-video Quizzes)
	\end{enumerate}
	\item These tagged elements cannot be accessed by the students having the default access to Bodhitree, i.e. students having access only to the free content.
	\item Several elements can be tagged under one tag, for example, a video titled \textit{``HTML 5 Intro"} and a document titled \textit{``CSS \& Styles"} can be tagged as \textit{``P1"}. So, such elements come under a group which is denoted by a tag. The access to that particular tag denotes the access provided to those group of elements.
	\item A student not having access to a particular tagged element can still see the title and the link to open that element, but when he tries to access the element, a dialogue box is displayed which informs him that he does not have access to that element.
	\item In case of video markers: If the video markers are tagged and the student does not have access to that particular tag, then the video will be shown as a plain MP4 video without any in-video quizzes or information markers.
	\item The instructor can upload a CSV file which contains the username's of the students that are currently registered on Bodhitree followed by the tags that the instructors want the students to have access to. A sample CSV file is shown below:
	
	\begin{center}
		\begin{tabular}{|c|c|c|c|}
		\hline \rule[-2ex]{0pt}{5.5ex} stud1 & P1 & P2 & P3 \\ 
		\hline \rule[-2ex]{0pt}{5.5ex} stud2 & P1 &  &  \\ 
		\hline \rule[-2ex]{0pt}{5.5ex} stud3 & P3 &  &  \\ 
		\hline 
		\end{tabular} \\
		\vspace{0.2cm}
		\textit{Table 2}: Sample CSV file for uploading the students access tags
	\end{center}
	
	\item The students who have access to a particular tag can access all the elements that are marked by that tag.
\end{enumerate}

\subsection{System Design}

%\subsection{Future Work}