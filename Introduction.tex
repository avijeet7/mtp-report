\section{Introduction}
%\par A massive open online course (MOOC) is an online course aimed at unlimited participation and open access via the web. In addition to traditional course materials such as filmed lectures, readings, and problem sets, many MOOCs provide interactive user forums to support community interactions between students, professors, and teaching assistants.\\
%\par Blending learning comes from the concept of flipped classrooms which refers to learning at one’s own pace, time and place. When used as a supplement to classroom teaching rather than as a replacement for it, MOOCs can certainly strengthen academia. They can provide a more prominent voice to the best teachers and are a means to close the gap between the most privileged learners and the underprivileged.\\
%\par Assessment and feedback are important factors for the success of MOOCs. Automation in assessment of quizzes must be well designed to provide well formed feedback to the user that can guide learning. The importance of the user interface elements along with several social tools such as collaborative discussions, notifications, and video-conferencing  also influences the learning experience.\\
%\par A number of challenges, including questions about the hybrid education, plagiarism, certification, completion rates, and innovation beyond traditional learning models exist which need to be addressed. Applying learning analytics tools for Technology Enhanced Learning provides a new platform for research.\\
%\par In a developing country like India, education plays a vital role in eradicating unemployment, which is a major concern. MOOCs are in an evolving stage in India where a few Indian Universities and Institutes have taken the initiative for their deployment.

\hspace{0.45cm} Bodhitree platform acts as a host to several courses which are offered as Small Private Online Courses (SPOCs) and managed by different instructors who set the content, exams and assignments of the course. The content in each course consists of several \textit{concepts} which can be viewed as chapters to a book. There are several \textit{learning elements} in the concepts which are in the form of videos, quizzes or documents.

\par Currently, the instructor does not have any control over what learning elements he can allow individual students to access. We have designed and implemented a module named \textit{Access Control} that allows the instructor to offer differential access to the learning elements to individual students. By default, the elements are \textit{Free} which means all the students can access the elements. The instructor can tag individual elements so that they are inaccessible to regular users and then provide access to individual users for those elements.

\par Several exams take place in the courses that are offered on Bodhitree. The exams may consist of online quizzes, programming assignments or offline exams. The grades given to these exams are displayed to the students by using means other than the Bodhitree platform such as Google Spreadsheets, etc. It is necessary that the Bodhitree platform provides an interface for the instructor to upload the marks of the students of some or all the exams that take place in the course and then the students are able to view the marks on Bodhitree itself. This functionality is offered by the \textit{Marks module}, which enables the instructor to upload a CSV file containing the marks of all the students along with their usernames, which then enables each student to view his/her own marks on Bodhitree.

\par Each course offered on Bodhitree consists of several concepts. It may occur that certain concepts must be completed before the beginning of certain other concepts. The instructor must address such relations and provide a guideline for the students to proceed in a course. Such relations can be mapped in the form of a directed graph which give a clear idea of what path to follow in order to have an optimal learning experience in the course. The graph prerequisite module aims to produce a guideline to the users by displaying a directed graph which shows the concepts of the course in the form of nodes and the dependencies amongst the concepts as edges of the nodes. The users can then proceed in the course by following the directions suggested by the prerequisite graph.