\section{Introduction}
%\par A massive open online course (MOOC) is an online course aimed at unlimited participation and open access via the web. In addition to traditional course materials such as filmed lectures, readings, and problem sets, many MOOCs provide interactive user forums to support community interactions between students, professors, and teaching assistants.\\
%\par Blending learning comes from the concept of flipped classrooms which refers to learning at one’s own pace, time and place. When used as a supplement to classroom teaching rather than as a replacement for it, MOOCs can certainly strengthen academia. They can provide a more prominent voice to the best teachers and are a means to close the gap between the most privileged learners and the underprivileged.\\
%\par Assessment and feedback are important factors for the success of MOOCs. Automation in assessment of quizzes must be well designed to provide well formed feedback to the user that can guide learning. The importance of the user interface elements along with several social tools such as collaborative discussions, notifications, and video-conferencing  also influences the learning experience.\\
%\par A number of challenges, including questions about the hybrid education, plagiarism, certification, completion rates, and innovation beyond traditional learning models exist which need to be addressed. Applying learning analytics tools for Technology Enhanced Learning provides a new platform for research.\\
%\par In a developing country like India, education plays a vital role in eradicating unemployment, which is a major concern. MOOCs are in an evolving stage in India where a few Indian Universities and Institutes have taken the initiative for their deployment.