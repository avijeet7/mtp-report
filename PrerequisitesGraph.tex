\section{Prerequisites Graph Module}

\subsection{Problem Statement}

\hspace{0.45cm} There are several courses offered on the Bodhitree platform. Each of these courses consist of several concepts. These concepts may or may not be related, but if they are related then there must be some sort of mapping between the concepts so as to have the best possible learning experience.
\par There should be a pattern that can be set by the instructor, as a pathway for the students to follow while they are taking up the course to ensure the best learning. Also, it is necessary to address those concepts in which a certain level of proficiency is required for a student to move on the a related concept.
\par The prerequisites graph module handles that task of representing such mappings in the form of a directed graph which presents the mapping amongst the several concepts that are present in a course.
\par It also add the links to the prerequisites to a concept on the main page of each concept, so the the user can navigate to the corresponding prerequisite before attempting to learn that concept.

\subsection{Specifications}

\begin{enumerate}
	
	\item The instructor while adding the concept should also specify the prerequisites for that particular concept. These prerequisites must then be shown on the concept page.
	
	\item An initial directed graph must be generated by using the prerequisites of the concepts which shows the mappings between the concepts and their prerequisites and which acts as a guide for the students to learn the concepts of the course in a particular manner.
	
	\item The Bodhitree platform must allow the instructor to be able to modify the prerequisite graph by adding/removing nodes and edges or changing the mappings of the graph. The final graph which the instructor decides should then be shown to the students who have taken up the course.

\end{enumerate}

\subsection{System Design}

\begin{enumerate}
	\item Several graph generation libraries were used to experiment and test the initial version of the graph that is to be produced. Several functions were implemented to generate the nodes and edges of the graph to be generated in the format required for the libraries. Some of the libraries that used are:
	\begin{enumerate}
		\item Graphviz: A python library which generated an image of the final graph in JPEG or PNG format. No interactivity was offered using this library.
		\item Arbor.js: It is a javascript library which generated interactive graphs which were dynamically created when a student opened the graphs page. It allowed to interact with the nodes, and information of the content that the node maps to was also displayed in an information box.
		\item GoJS: It allowed the creation of an initial graph, which offered interactivity in a sense that the user can add/remove the nodes and edges and modify them accordingly.
	\end{enumerate}

	\item The prerequisites of each concept are retrieved from the database when a concept loads and they are displayed as a list on the concept page. The user can then click on them to navigate to that concept.
\end{enumerate}

\subsection{Future Work}

\begin{enumerate}

	\item There is a need to develop a design in which the instructor can upload an initial version of the prerequisite graph consisting of the nodes and the edges in a particular format, which allows modifiability.
	
	\item An interface must be provided on the Bodhitree platform which generated the prerequisite graph by fetching data from the database, allowing the instructor to restructure the graph according to his wish and save it as a final prerequisite graph for the course that is to be displayed to the users.
	
	\item Allowing interactivity to the nodes will allow display of additional information like the concept details, etc.
	
	\item Interactivity of the users with the nodes of the graph along with dynamically generating the graph will further allow the possibility to have a graphical interface to show the progress of each individual student in the form of a coloured/labelled directed graph.

\end{enumerate}