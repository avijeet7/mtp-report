\vspace{2in}
\begin{abstract}
%The dominance of Internet in our lives have given an opportunity to deliver quality education to the masses throughout the world in the form of MOOCs (Massive Open Online Courses). MOOCs are scalable and free and are a means to bridge the gap between the most privileged and the underprivileged learners. However, several challenges such as scalability, high drop rates, and meeting various demands of the learners exist which affect the efficiency of learning. Certain guidelines have been proposed which should be followed while creating a MOOC in order to make it successful. Learning analytics can be used to further improve their effectiveness. The Indian education system faces major challenges which can be counteracted by supplementing it with MOOCs. This seminar is a survey of the MOOC platform. A detailed explanation and classification of the MOOC components and several leading MOOC providers is made and several techniques are introduced to increase the overall effectiveness of education.
\vspace{0.5cm}
The dominance of Internet in our lives have given an opportunity to deliver quality education to the masses throughout the world. A lot of educational platforms have been developed to help students understand any concept easily.  One such platform is Bodhitree, which is an educational website used in IIT Bombay. Although Bodhitree is a full fledged educational website, several additional features are yet to be added to it so as to improve upon the experience the students as well as the instructors have while using this platform.  We have added several features to the Bodhitree platform such as the upload and display of student’s marks for a particular course and implementing access control for the learning elements such as videos, quizzes, documents etc.  In addition to that, we are working on adding a directed graph to Bodhitree, which will show a mapping of the prerequisites to the concepts whose knowledge is essential before learning a new concept.

\end{abstract}